\documentclass{beamer}

% Definições de cores
\usepackage{xcolor}
\definecolor{mygreen}{RGB}{0, 128, 0}

% Configurações de tema e cor
\usetheme{default}
\setbeamercolor{title}{fg=mygreen}
\setbeamercolor{frametitle}{fg=mygreen}
\setbeamercolor{normal text}{fg=black, bg=white}

% Configurações de fontes
\setbeamerfont{title}{size=\LARGE}
\setbeamerfont{frametitle}{size=\LARGE}
\setbeamerfont{normal text}{size=\large}

\begin{document}

% Slide de título
\begin{frame}
    \title{OpenCV e NumPy}
    \subtitle{Introdução e Funcionalidades}
    \author{Luis Eduardo Lima}
    \date{\today}
    \titlepage
\end{frame}

% Slide sobre OpenCV
\begin{frame}
    \frametitle{O que é OpenCV?}
    \begin{itemize}
        \item OpenCV é uma biblioteca de visão computacional e processamento de imagem de código aberto.
        \item Contém mais de 2500 algoritmos otimizados para uma ampla gama de tarefas.
    \end{itemize}
\end{frame}

% Slide sobre Funcionalidades do OpenCV
\begin{frame}
    \frametitle{Funcionalidades do OpenCV}
    \begin{itemize}
        \item Processamento de Imagem: Filtros, transformações, manipulação de cores.
        \item Análise de Vídeo: Captura de vídeo, processamento em tempo real, rastreamento de objetos.
        \item Reconhecimento de Padrões: Detecção de faces, identificação de objetos, reconhecimento de texto.
        \item Visão 3D: Estimação de movimento, reconstrução 3D, mapeamento estéreo.
        \item Aprendizado de Máquina: Integração com modelos de aprendizado de máquina.
    \end{itemize}
\end{frame}

% Slide sobre Implementação e Recursos do OpenCV
\begin{frame}
    \frametitle{Implementação e Recursos do OpenCV}
    \begin{itemize}
        \item Implementado principalmente em C++, com interfaces para Python, Java e MATLAB/OCTAVE.
        \item \textbf{Site Oficial}: \url{https://opencv.org/}
        \item \textbf{Documentação}: \url{https://docs.opencv.org/}
        \item \textbf{GitHub}: \url{https://github.com/opencv/opencv}
    \end{itemize}
\end{frame}

% Slide sobre NumPy
\begin{frame}
    \frametitle{O que é NumPy?}
    \begin{itemize}
        \item NumPy é uma biblioteca fundamental para computação científica em Python.
        \item Fornece um objeto de array multidimensional de alta performance.
    \end{itemize}
\end{frame}

% Slide sobre Funcionalidades do NumPy
\begin{frame}
    \frametitle{Funcionalidades do NumPy}
    \begin{itemize}
        \item Array Multidimensional: Estrutura de dados central de NumPy é o array N-dimensional, \texttt{ndarray}.
        \item Funções Matemáticas: Suporta operações aritméticas e funções matemáticas complexas em arrays.
        \item Álgebra Linear: Funções para decomposição de matrizes, sistemas lineares, transformações.
        \item Transformações de Fourier: Implementações rápidas de transformações de Fourier.
        \item Ferramentas Estatísticas: Cálculo de média, mediana, desvio padrão e outras funções estatísticas.
    \end{itemize}
\end{frame}

% Slide sobre Implementação e Recursos do NumPy
\begin{frame}
    \frametitle{Implementação e Recursos do NumPy}
    \begin{itemize}
        \item Implementado em Python, com operações de baixo nível otimizadas em C.
        \item \textbf{Site Oficial}: \url{https://numpy.org/}
        \item \textbf{Documentação}: \url{https://numpy.org/doc/}
        \item \textbf{GitHub}: \url{https://github.com/numpy/numpy}
    \end{itemize}
\end{frame}

% Slide sobre Implementação Combinada
\begin{frame}
    \frametitle{Implementação Combinada}
    \begin{itemize}
        \item NumPy facilita a criação e manipulação de arrays.
        \item OpenCV usa esses arrays para processar e analisar imagens.
        \item \textbf{Exemplo}:
        \begin{verbatim}
import cv2
import numpy as np

lower_blue = np.array([100, 150, 50])
upper_blue = np.array([140, 255, 255])
        \end{verbatim}
    \end{itemize}
\end{frame}

% Slide sobre Recursos Adicionais
\begin{frame}
    \frametitle{Recursos Adicionais}
    \begin{itemize}
        \item \textbf{Cursos Online}: Coursera, Udacity, edX, Udemy.
        \item \textbf{Livros}: "Learning OpenCV" por Gary Bradski e Adrian Kaehler, "Python for Data Analysis" por Wes McKinney.
        \item \textbf{Comunidade}: Stack Overflow, fóruns, grupos de estudo, conferências como PyCon.
    \end{itemize}
\end{frame}

% Slide de conclusão
\begin{frame}
    \frametitle{Conclusão}
    \begin{itemize}
        \item OpenCV e NumPy são ferramentas poderosas para visão computacional e análise de dados científicos.
        \item Utilizar essas bibliotecas pode simplificar e otimizar o desenvolvimento de projetos complexos.
    \end{itemize}
\end{frame}

\end{document}

